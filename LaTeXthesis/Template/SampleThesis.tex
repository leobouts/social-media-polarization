% !TEX TS-program = XeLaTeX
%
% Περάστε στο πακέτο cseuoi-thesis τις κατάλληλες επιλογές για τη διατριβή σας
%
%\documentclass[gr,phd]{cseuoi-thesis} % Διδακτορικό (στα Ελληνικά)
\documentclass[en,msc,systems]{cseuoi-thesis} % Μεταπτυχιακό με εξειδίκευση στα Υπολογιστικά Συστήματα (στα Αγγλικ)ά
%\documentclass[en,msc,theory]{cseuoi-thesis} % Μεταπτυχιακό με εξειδίκευση στη Θεωρία Επιστήμης Υπολογιστών (στα Αγγλικά)
%\documentclass[en,msc,software]{cseuoi-thesis} % Μεταπτυχαικό με εξειδίκευση στο Λογισμικό (στα Αγγλικά)
%\documentclass[en,msc,scicomp]{cseuoi-thesis} % Μεταπτυχιακό με εξειδίκευση στους Επιστημονικούς Υπολογισμούς (στα Αγγλικά)
%\documentclass[en,msc,techapps]{cseuoi-thesis} % Μεταπτυχιακό με εξειδίκευση στις Τεχνολογίες - Εφαρμογές (στα Αγγλικά)


% 
% Συμπληρώστε τα στοιχεία σας στις παρακάτω εντολές (αφαιρώντας το \colorbox{gray}{})
%
\titleEn{Reducing Polarization in Social Media}
\authorEn{Leonidas Boutsikaris}
\arthro{τον}
\aitiatiki{}
\dateEn{February 2021}
\advisorEn{Panayiotis Tsaparas, Assistant Professor}



% 
% Εξεταστική επιτροπή για ΜΔΕ
%
\MSCexaminer{Panayiotis Tsaparas}{Associate Professor}%
    {Department of Computer Science and Engineering, University of Ioannina (Supervisor)}
\MSCexaminer{? ?}{Associate Professor}%
    {Department of Computer Science and Engineering, University of Ioannina }
\MSCexaminer{? ?}{Associate Professor}%
    {Department of Computer Science and Engineering, University of Ioannina }


% Πακέτο για την εμφάνιση περιθωρίων (χρήσιμο για την εύρεση overfull boxes)
%\usepackage{showframe}

% Πακέτο για τη διατήρηση των floats (εικόνες κ.α.) εντός των ενοτήτων
%\usepackage[section]{placeins}



\begin{document}

% Σελίδες χωρίς αρίθμηση
\pagenumbering{gobble}

% Εκτύπωση της σελίδας τίτλου
\maketitle

% Εκτύπωση της σελίδας με τις επιτροπές
\makecommittees

% Αρχικοποίηση του minitoc
\dominitoc[n]

%\include{FrontMatter/Dedication} % Προαιρετικό
%\include{FrontMatter/Acknowledgements} % Προαιρετικό


% Σελίδες με αρίθμηση i, ii, iii, iv, ...
\pagenumbering{roman}

% Περιεχόμενα
\pdfbookmark{\contentsname}{contents} % hyperref
\tableofcontents

% Κατάλογος Σχημάτων
%\addstarredchapterc{\listfigurename} % minitoc
%\listoffigures

% Κατάλογος Πινάκων
%\addstarredchapterc{\listtablename} % minitoc
%\listoftables

% Κατάλογος Αλγορίθμων
%\addstarredchapterc{\listalgorithmname} % minitoc
%\listof{algorithm}{\listalgorithmname}

%\include{FrontMatter/Glossary} % Προαιρετικό

% Περίληψη και εκτεταμένη περίληψη
%\chapter*{\abstractname}
\addstarredchapter{\abstractname} % minitoc

\noindent We know for a fact that opinions are formed through social interactions. Online communities offer public access to social disputes on controversial matters that allow the study and moderation of them. The majority of studies in social networks are based on the Friedkin-Johnsen model.
\\
\\
Users of online communities are receiving biased information that amplify their own viewpoints. This creates a fragmented community and users interact only with individuals that hold the same opinions. In this thesis we use a metric, the polarization index, to measure the polarization of a social graph.
\\
\\We try to reduce the polarization by connecting individuals. We propose new social connections between different and extreme opinions by following the intuition of the Friedkin-Johnsen model. 
\\
\\
Finally, probabilities are adopted to our heuristic algorithms that now make selections based on how probable is the acceptance of a recommendation of a new social connection. 

\bigskip

%\include{FrontMatter/ExtendedAbstract}


% Σελίδες με αρίθμηση 1, 2, 3, 4, ...
\pagenumbering{arabic}


% Εισαγωγή των κεφαλαίων
\chapter{Introduction and the Theory Behind Polarization}
\label{ch:Introduction}


\section{Introduction}
\label{sec:Objectives}

Real world events such as Brexit and the 2016 U.S. presidential elections give us a clear hint about the polarization our society is witnessing. 
\\
\\
Social media polarization has a strong effect on politics, opinion formation and how people interact with each other in a society. Users of social media are now receiving biased information that amplify their own viewpoints. 
\\
\\
Enclosed in their filter bubble, they will ignore everyone else and only acknowledge opinions that fit their own reality. In combination with fake news a malicious entity could use social media as a tool to polarize certain groups of people for their own interest. 
\clearpage

\noindent Our goal is to decrease the polarization by proposing new social connections. These additions are computed using heuristic algorithms.
\\
\\
\noindent In a real world setting, new social connections are not always accepted.  For example we would not accept friend requests from people we barely know.
 This is relevant with link prediction. Link prediction is the problem of predicting the existence of a link between two entities in a network in the future. For example the "People you may know" section on Facebook.		
\\		
\\		
Link prediction algorithms are based on how similar two different nodes are, what features they have in common, how are they connected to the rest of the network or how many other nodes are connected to a single node. 		
Link prediction is also used in recommendation systems and  information retrieval.
\\
\\
By using the acceptance probabilities of a link prediction model we can define the expected decrease of the polarization. For computing these probabilities we will use graph embeddings.

\section{Motivation}
\label{sec:motivation}

Polarization describes the division of people into two contrasting groups or sets of opinions or beliefs. The term is used in various domains such as politics and social studies.
In social media settings, users tend to join communities of like-minded individuals.
\\
\\
In these settings the opinions of the users are amplified and reinforced by the continuous  communication and recycling of the same view. These communities are referred to as echo chambers.
\\
\\
Inside an echo chamber users can easily find information that reinforces their existing opinion without being exposed to opposing views.
Echo chambers can be created where information is exchanged, whether it’s online or in real life. 
\\
\\
On social media almost anyone can quickly find like-minded people and countless news sources. This has made echo chambers far more numerous and easy to fall into. 
\\
\\
Echo chambers online are referred to as filter bubbles. Filter bubbles are created by algorithms that keep track of the online behaviour of a user such as search histories, shopping activity and many more.
\\
\\
Social media will then use those algorithms to show content that is similar to what the user is already aligned with. This can lead users to adopt a more extreme version of their opinions. 


\section{Thesis Outline}
\label{sec:outline}

We begin by exploring the Friedkin and Johnsen Model. This model uses the terms of internal and external opinion. By repeated averaging and combining these two values we can get the opinion vector of the graph.
This is a vector that contains information about the opinions of the whole network. Then, by defining and computing the polarization index we can get an image of the social graph.
\\
\\
We then proceed and define our two problems. First, want to find the best $k$ edges that will lead to the greatest reduction of the polarization index. The second problem incorporates the acceptance probabilities.
We also observe that the addition of new edges between opposing opinions will not necessary decrease the polarization index  and prove it with a counter-example.
\\
\\
Our heuristics are based on the intuition that our model has the biggest decrease when we connect different and extreme opinions and classify them in two categories. 
\\
\\
In these two categories the heuristics do or do not recompute the graph structure after the addition of an edge. This is derived from the fact that when adding an edge to the network the structure of the graph changes.

\clearpage


\noindent The heuristics that consider network changes are the $Greedy$ the $FTGreedy$ and the $Expressed Opinion$. These three are then modified into a batch version that does not consider network changes.
\\
\\
We continue by using Graph Embeddings and the $Node2Vec$ algorithm to compute acceptance probabilities. We use these probabilities in a modified version of our heuristics to compute the expected decrease of the polarization index.
\\
\\
Our heuristics are applied in 6 datasets of various topics and compared with each other. The Greedy heuristics cannot run on graphs that contain a lot of nodes due to time limitations.

\section{Roadmap}
\label{sec:roadmap}

At chapter 2 we look into the related work around polarization and decreasing polarization. We see how polarization is measured, the relation of polarization and random walks and how polarization can be combined with disagreement and conflict. 
\\
\\
At chapter 3 we define the Friedkin and Johnsen model and the polarization index. We also define our 2 problems, the $k-Addition$ problem and the $k-Addition-Expected$. Then, we proceed to explore the monotonicity of the polarization index.
\\
\\
At chapter 4 we define our heuristics and at chapter 5 we continue by including acceptance probabilities in them.
\\
\\
At chapter 6 we take a look at our datasets and obtain the results of the polarization decrease for each heuristic in each dataset. 





\chapter{Related Work}
\label{ch:Instructions}

We will now present some work that is related to the work in this thesis. 

\section{Opinion models}
\label{sec:opinionsModels}

How people form their opinions has long been the subject of research in the field of social sciences. Models of opinion formation and dynamics are being used  by computer scientists to explore and quantify polarization, conflict and disagreement on social networks.
\\
\\
Opinion models study these quantities and how they change by manipulating the opinions and by changing the network structure of a set of nodes of the social graph.

\clearpage
\noindent Many of these models are based on the influence that goes with social interaction and the Friedking-Johnsen model is a very popular one.

\section{Measuring the Polarization of a Network}
\label{sec:Submission}

In this paper the polarization index is defined. The direct link between the Friedkin-Johnsen model and random walks is also explored. 
\\
\\
Two problems are introduced, the $ModerateInternal$ and the $ModerateExpressed$. When moderating opinions a small set of nodes $T_s$ is being set to zero, in each problem, as their names suggests, internal or external opinions are set to zero. Two algorithms are proposed for the $ModerateInternal problem$. 
\\
\\
A greedy algorithm that finds the set $T_s$ of nodes iteratively according to the biggest decrease it causes and the Binary Orthogonal Matching Pursuit (BOMP) algorithm. 
\\
\\
For the $ModerateExpressed$ problem the same greedy algorithm is used.
\cite{tsapMatakosTerzi}


\section{Polarization and Disagreement}
\label{sec:polarizationDisareement}

Another way of looking at polarization is by combining it with disagreement. The main problem of minimising polarization and disagreement lies in the opinions of each user and how targeted ads and recommendations influence their opinions. 
\\
\\
Considering the disagreement in combination with polarization a network can choose how to respond in different situations. Their recommendation system could choose between keeping the disagreement low or exposing users to radically different opinions. 
\\
\\
There are situations that this optimisation can reduce the overall polarization-disagreement in the network by recommending edges in different parts of the network than the ones that agree with the human confirmation bias. 

\clearpage

\noindent In this paper the disagreement of an edge is defined as the squared difference between the opinions of this edge. The total disagreement is defined as the sum of the disagreements and 
polarization is measured as a deviation from the average with the standard definition of variance.
\\
\\
The polarization-disagreement index is defined as follows $I_{G,s} = P_{G,s} +D_{G,s}$ which is the sum of the total disagreement and polarization. The objective is to minimise this index. This objective is addressed as an optimization problem.
\cite{musco}


\section{Quantifying and Minimizing Risk of Conflict in Social Networks}
\label{sec:riskOfConflict}

This paper addresses the main problem in the Friedkin-Johnsen model metrics. The external opinion of a user, which by definition is hard to measure, combined with the internal opinion which is impossible to be measured. 
\\
\\
Another problem occurs in the editing of the social graph. When the social graph is edited its is done in a way that minimises the conflict of a certain social issue. This can lead to an increased conflict of one or more social issues inside the network.
\\
\\
Chen, Lijffijt and De Bie still use the Friedkin-Johnsen model to evaluate the network conflict but the quantifications depend only on the network topology in a way that the conflict can be reduced over all issues. 
\\
\\
Worst-case(WCR) conflict risk and average-case conflict risk(ACR) are defined to represent two separate problems, how the network can be minimised in the worst case or in the average case scenario by altering the social graph. 
\\
\\
 These problems consider the measures of internal conflict, external conflict, and controversy. Internal conflict ($ic$) measures the difference of the internal and the expressed opinion of a user. $ic = \sum_i{(z_i-s_i)^2}.$
 \clearpage
 
\noindent These measures are not independent. Reducing one of them results in the increase of another. This leads to the conservation law of conflict.
 \\
 \\
There are two methods of minimising the conflict of the network for each of the ACR and WCR problems. One is a gradient method that  considers deleting and adding edges simultaneously and the other is a descent method that suggests deleting or adding a single edge. \cite{chen}


\section{Reducing Controversy by Connecting Opposing Views}
\label{sec:reducing}

Garimella et al. rely on a measure of controversy that is shown to work reliably in multiple domains in contrast with other measures that focus on a single topic. The controversy measure consists of the following steps:

\begin{enumerate}
  \item Given a topic $t$ they create an endorsement graph $G=(V,E)$. This graph represents users who have generated content relevant to $t$. For example hashtags of a user.
  
  \item The nodes of this graph a re partitioned in two disjoint sets $X$ and $Y$. The partition is obtained using a graph-partition algorithm.
  
  \item The last step, is computing the controversy measure through a random-walk, thus creating the controversy score $RWC$. This score is defined as the difference of the probability that a random walk starting on one side of the partition will stay on the same side and the probability that the random walk will cross to the other side. A personalised PageRank is used where the restart probabilities are set to a random vertex of each side.
\end{enumerate}
\vspace{4pt}
Garimella et al. states that real graphs often have a star-like structure. Small number of highly popular vertices have a lot of incoming edges. These nodes can be seen as thought leaders and their followers. It is shown that connecting the high degree vertices minimises the $RWC$ score.
\\
\\
Probabilities are also incorporated in the sense that a new edge addition may be not accepted by the user. \cite{garimella}

\chapter{Premilinaries and Problem Definition}
\label{ch:premAndDef}


\section{The Friedkin and Johnsen Model}
\label{sec:prem}

The Friedkin-Johnsen model is a very popular extension of the DeGroot's model. This model uses information about the opinion of the user assuming there is an internal and external opinion. The internal opinions cannot change and is the specific opinion of an individual for a certain matter. On the other hand the expressed opinion is influenced by social interactions.
\\
\\
The expressed opinion of a user is computed as a weighted average of the external opinions of the neighbourhood of the user, for example, the opinions of the users friend list or the accounts the user follows.
\\
\\
The opinions of the users are stored in a vector. This vector is a metric for the whole social graph and can give us insights about its current situation. 
\clearpage


\noindent The vector values range from [-1,1]. Values closer to the range limits indicate bigger polarization. Polarized graphs create groups of nodes that are strongly connected with each other.
\\
\\
\noindent Let $G = (V,E)$ be a connected undirected graph representing a network. Let $z$ be the vector of expressed opinions  for the whole network. Each value  of the vector represents a node and can be computed with the opinion-formation model of Friedkin and Johnsen as follows. 

\begin{equation} 
	z_i = \frac{w_{ii}*si + \sum_{j \epsilon N(i) }{w_{ij}*z_j}} {w_{ii} + \sum_{j \epsilon N(i) }{w_{ij}}} 
\end{equation} 
\\

\noindent Where $s_i$ denotes the internal and $z_i$ the expressed opinion of a user. The internal opinion of a user corresponds to the views that inherently holds for a controversial topic while the expressed refers to the views that the user shares on a social network with his friends. 
\\
\\
The length of the opinion vector $||z|| ^2$ measures  the polarization of the network. To make the polarization  independent of its network we can  normalize it  by dividing  it with the length of the vector $z$. 
\\
\\
An equivalent way of obtaining the vector $z$ from a graph is the following: if $L$ is the laplacian matrix of a graph $G=(V,E)$, and $I$ is the identity matrix, then $z=(L+I)^{-1}S$ \cite{bindel}. 

\section{Measuring the polarization}
\label{sec:meas}

\noindent We use the definition of the polarization index by Matakos et al. The polarization is measured by its distance from a neutral opinion. 
\\
\\
We can quantify this with the length of the vector of the second norm $L_{2}^2$ \cite{tsapMatakosTerzi}.
\clearpage


\begin{equation}
	\pi(z) = ||z||_{2}^2
\end{equation}
\\
This value can be independent of the network if we normalize it by dividing with the size of the graph.

\section{Problem Definition}
\label{sec:problemDef}

\noindent Let $G = (V,E)$ be a connected undirected graph representing a network. Let $z$ be the vector of expressed opinions  for the whole network and $\pi(z) = ||z||_{2}^2$ the polarization index of this social graph.
\\
\\
\noindent Problem 1 [k-Addition]. Let $C \subseteq	V \times V$ a set of edges that are not in the graph. We want to find a subset of $S \subseteq C$ of $k$ edges whose addition to a graph $G$ leads to the greatest reduction of $\pi(z)$.


\subsection{Including probabilities into the problem}
\label{sec:probabilityProbDef}

Problem 1 is trying to find edges that will minimize the polarization index. We must not take for granted that these edges will be accepted. 
For example a social media user could reject a new follow/friend request. This leads us to consider additions with the expectation of being accepted. 
\\
\\
The expected polarization can be defined as follows.
\\
\\
\begin{equation}
	E[\pi(z)]= P(u,v) * Val
\end{equation}
\\
\noindent Where $P(u,v)$ is the probabilty of $u$ and $v$ forming an edge and $Val$ is a value specific to our heuristic algorithms. For example $Val$ can be the polarization decrease after the edge addition of $u$ and $v$ or the absolute distance of their opinions.
\clearpage


\noindent Problem 2 [K-Addition-Expected]. Given a graph $G=(V,E)$ and an integer $k$, we want to find a set of $k$ edges $E′ \subseteq V×V \ E$ that when added to $G$ creates a new graph $G' = (V,E \cap E')$ so that the expected polarization score $E[\pi(z)]$ is minimized.


\section{Monotonicity of the Problem}
\label{sec:monotonicity}
\vspace{20pt}
We observe that $\pi(z)$ is not monotone with respect to the edge additions. This means that adding an edge will not necessarily decrease the polarization index. 
\\
\begin{lemma}
The polarization index does not necessarily decrease after an edge addition between opposing views.
\end{lemma}

\vspace{20pt}
\noindent We will  show this with a counter example. In the network~\ref{fig:p5} nodes 0, 2 and 3 have a value of $s_i=-1$, and nodes 2 and 4 have a value of $s_i=+1$. For both examples we assume that $w_{ii}=w_{ij}=w_{ji}=1$ and $n$ the number of nodes.
\\
\\
We will now compute the polarization index of the original graph
\\
\\
\begin{figure}[h]
	\centering
	\begin{subfigure}[t]{0.3\textwidth}
		\centering
		\includegraphics[height=0.15\textheight]{Figures/p5A}
		\caption{}
		\label{subfig:monotonicityA}
	\end{subfigure}
	\hfill
	\begin{subfigure}[t]{0.3\textwidth}
		\centering
		\includegraphics[height=0.15\textheight]{Figures/p5B}
		\caption{}
		\label{subfig:monotonicityB}
	\end{subfigure}
	\vspace{40pt}
	\hfill
	\caption{Edge addition between opposed opinions.}
	\label{fig:p5}
\end{figure}
\\
\\

\clearpage

\begin{equation}
	\begin{aligned}
		(L+I)^{-1}s=z=
		\left(\begin{matrix}
		\frac{-27}{55} \\
		\frac{1}{55} \\
		\frac{-5}{11} \\
		\frac{-21}{55} \\
		\frac{17}{55}
		\end{matrix}\right),
		\qquad \qquad
		\pi(z)= 0.13785123966
	\end{aligned}
\end{equation}
\\
We will now compute the polarization index after the addition of the edge $1\rightarrow3$.

\begin{equation}
	\begin{aligned}
		(L+I)^{-1}s=z=
		\left(\begin{matrix}
		\frac{-53}{99} \\
		\frac{-7}{99} \\
		\frac{-5}{11} \\
		\frac{-29}{99} \\
		\frac{35}{99}
		\end{matrix}\right),
		\qquad \qquad
		\pi(z) = 0.14180185695
	\end{aligned}
\end{equation}
\\
\\
We can see an increase of the polarization index after adding this particular edge. This example was discovered after brute-forcing different graph topologies with different combinations of opinion values.
\clearpage


\chapter{Algorithms}
\label{ch:algorithms}



\section{Intuition}
\label{sec:intuition}

To solve this problem we have to evaluate all possible edge combinations. Even for greedy heuristics we need to limit the edge candidates. The algorithm considers high-degree vertices of each polarized community. This is motivated by the tendency that social graphs resemble a star-graphs. That means a small number of nodes with high popularity are connected with a lot of low popularity nodes. Garimella Et Al \cite{garimella} uses this star like topology to count the decrease of the polarization. While the intuition behind this process is the same we use a different metric. Garimella Et Al uses a random-walk controversy score while we use the polarization index$\pi(z)$.
\\
\\
Let $G$ be our social network graph. In the figure~\ref{fig:starA} bellow we can see how it resemble a star like shaped network. Nodes $0-4$ have an internal value of $Z_i = 1$ and nodes $5-9$ have an internal value of $Z_i=-1$. Now we are going to compute the polarization index for the network $G$. 
\\

\begin{figure}[t]
	\centering
	\includegraphics[width=0.65\textwidth]{Figures/starA}
	\caption{A star like graph with two communities.}
	\label{fig:starA}
\end{figure}

\begin{equation}
Z_0 = \frac{1*(1) + 4*(1)}{1 + 4} = \frac{5}{5} = 1
\end{equation}

\begin{equation}
Z_1,Z_2,Z_3,Z_4 = \frac{1*(1) + 1*(1)}{1 + 1} = \frac{2}{2} = 1
\end{equation}

\begin{equation}
Z_5 = \frac{1*(-1) + 4*(-1)}{1 + 4} = \frac{-5}{5} = -1
\end{equation}

\begin{equation}
Z_6,Z_7,Z_8,Z_9 = \frac{1*(-1) + 1*(-1)}{1 + 1} = \frac{-2}{2} = -1
\end{equation}

\begin{equation}
||\pi(z)||^2 = \frac{\sqrt{10}}{10} = 0.316227766
\end{equation}
\\

There are four edge additions that we consider. Connecting the center nodes or the most popular nodes of each polarized community together by connecting node $0$ with node $5$ $(case 1)$. Connecting the center node of the one side of the polarized community with a non central node by connecting node $0$ with node $6$ $(case 2)$and in a respectively for the other polarized community node $5$ with node $4$ $(case 3)$. Finally connecting non central nodes together by connecting node $1$ with node $9$ $(case 4)$.
We will now compute the reduction of each case.


\begin{figure}[t]
	\centering
	\begin{subfigure}[t]{0.3\textwidth}
		\centering
		\includegraphics[height=0.15\textheight]{Figures/starB}
		\caption{}
		\label{subfig:starB}
	\end{subfigure}
	\hfill
	\begin{subfigure}[t]{0.3\textwidth}
		\centering
		\includegraphics[height=0.15\textheight]{Figures/starC}
		\caption{}
		\label{subfig:starC}
	\end{subfigure}
	\hfill
	\begin{subfigure}[t]{0.3\textwidth}
		\centering
		\includegraphics[height=0.15\textheight]{Figures/starD}
		\caption{}
		\label{subfig:starD}
	\end{subfigure}
	\begin{subfigure}[t]{0.3\textwidth}
		\centering
		\includegraphics[height=0.15\textheight]{Figures/starE}
		\caption{}
		\label{subfig:starE}
	\end{subfigure}
	\caption{Edge addition cases}
	\label{fig:edgeCases}
\end{figure}



$case1$

\begin{equation}
	\begin{aligned}
		&Z_0 = \frac{1*(1) + 4*(1)+1*(-1)}{1 + 5} = \frac{4}{6} = \frac{2}{3}\\
		&Z_1,Z_2,Z_3,Z_4 = \frac{1*(+1) + 1*(+1)}{1 + 1} = \frac{2}{2} = 1\\
		&Z_5 = \frac{1*(-1) + 4*(-1) +1*(+1)}{1 + 5} = \frac{-4}{6} =  -\frac{2}{3}\\
		&Z_6,Z_7,Z_8,Z_9 = \frac{1*(-1) + 1*(-1)}{1 + 1} = \frac{-2}{2} = -1\\
		&||\pi(z)||^2 = 0.298142397
	\end{aligned}
\end{equation}

$case2$
\begin{equation}
	\begin{aligned}
		&Z_0 = \frac{1*(+1) + 4*(1)+1*(-1)}{1 + 5} = \frac{4}{6} = \frac{2}{3}\\
		&Z_1,Z_2,Z_3,Z_4 = \frac{1*(+1) + 1*(+1)}{1 + 1} = \frac{2}{2} = 1\\
		&Z_5 = \frac{1*(-1) + 4*(-1)}{1 + 4} = \frac{-5}{5} = -1\\
		&Z_6 = \frac{1*(-1)+1*(-1) + 1*(+1)}{1+2}=-\frac{1}{3}\\
		&Z_7,Z_8,Z_9 = \frac{1*(-1) + 1*(-1)}{1 + 1} = \frac{-2}{2} = -1\\
		&||\pi(z)||^2 = 0.29249881291
	\end{aligned}
\end{equation}


$case3$
\begin{equation}
	\begin{aligned}
		&Z_0 = \frac{1*(+1) + 4*(1)}{1 + 4} = \frac{5}{5} = 1\\
		&Z_1,Z_2,Z_3 = \frac{1*(+1) + 1*(+1)}{1 + 1} = \frac{2}{2} = 1\\
		&Z_4 = \frac{1*(+1) + 1*(-1)}{1 + 2} = -\frac{-1}{3}\\
		&Z_5 = \frac{1*(-1)+4*(-1) + 1*(+1)}{1+5}=-\frac{4}{6}=-\frac{2}{3}\\
		&Z_6,Z_7,Z_8,Z_9 = \frac{1*(-1) + 1*(-1)}{1 + 1} = \frac{-2}{2} = -1\\
		&||\pi(z)||^2  = 0.29249881291
	\end{aligned}
\end{equation}


$case4$

\begin{equation}
	\begin{aligned}
		&Z_0 = \frac{1*(+1) + 4*(1)}{1 + 4} = \frac{5}{5} = 1\\
		&Z_1,Z_2,Z_3 = \frac{1*(+1) + 1*(+1)}{1 + 1} = \frac{2}{2} = 1\\
		&Z_4 = \frac{1*(+1)+1*(+1) + 1*(-1)}{1 + 2} = \frac{1}{3} \\
		&Z_6 = \frac{1*(+1) + 1*(-1)+ 1*(-1)}{1 + 2} = -\frac{1}{3}\\
		&Z_5 = \frac{1*(-1) + 4*(-1)}{1 + 4} = -\frac{5}{5} = -1\\
		&Z_7,Z_8,Z_9 = \frac{1*(-1) + 1*(-1)}{1 + 1} = \frac{-2}{2} = -1\\
		&||\pi(z)||^2 = 0.28674417556
	\end{aligned}
\end{equation}

Therefore, the maximum decrease is achieved by the addition of node $4\rightarrow6$. Even though real graphs do not match this case exactly, they often have a structure that resembles star-graphs in certain ways: a small number of highly popular vertices receive incoming edges from a large number of other vertices \cite{garimella}.

\section{Proposed Algorithm}
\label{sec:proposedAlgorithm}

The results from section~\ref{sec:intuition} makes us consider edge addition between low-degree vertices from each side of the polarized communities. The algorithm can be seen in the figure Algorithm~\ref{alg:algorithm}.

\begin{algorithm}[t]
	\caption{Minimization of the polarization index $\pi(z)$}
	\label{alg:algorithm}
	\begin{flushleft}
        		\textbf{INPUT:} Graph $G$, number of edges to add , $k$; $k1$, $k2$, low degree vertices in each of the polarized communities in $X$,$Y$ respectively.\\
        		\textbf{OUTPUT:} List of $k$ edges that minimize the polarization index $\pi(z)$
	\end{flushleft}
	\begin{algorithmic}[1]
		\STATE $Out \leftarrow Empty List;$
		\FOR {$i = 1:k1 \ do$}
		\STATE $Vertex \ u = X[i]$
		\FOR {$j= 1:k2 \ do$}
		\STATE $Vertex \ v = Y[i]$
		\STATE Compute $\pi(z)$, the decrease if the edge $(u,v)$ is added;
		\STATE Append edge $(u,v)$ to Out;
		\ENDFOR
		\ENDFOR
		\STATE $Sorted \leftarrow sort(Out)$ by $\pi(z)$ by decreasing order;
		\STATE Return top $k$ from $Sorted$
	\end{algorithmic}
\end{algorithm}



\chapter{Experiments}
\label{ch:experiments}

\section{Datasets}
\label{sec:ds}

In this section we consider datasets that are separated in two opposing communities. The information about the opinions of each member of this community is known. Thus, we can assign internal opinions -1 and 1 to the nodes depending on their community membership\cite{tsapMatakosTerzi}. 

\begin{enumerate}

  \item The Karate dataset, that represents the friendships between the members of a karate club at a US university. This network is split in two equal size polarized communities arround two rival instructors.
  
  \item The Books dataset that is a networkd of US politics books. These books were published near the 2004 presidential election and sold by Amazon.com . These Books are classified as "Liberal", "Conservative", or "Neutral".
  
  \item The Blogs dataset. A network of hyperlinks between online blogs on US politics.
  
\end{enumerate}
\section{Experiments with heuristics}
\label{sec:experimHeuristics}

 We evaluate the heuristic algorithms by comparing them with the naive. The goal is to validate that they minimize the polarization index in the same way the naive algorithm does but in less time. 


\begin{table}[htbp]
 \centering
 \caption{Heuristics algorithms comparison on the Karate}
 \label{tab:heuristicsKarate}
 \begin{tabular}{|c |l| c | c | c | c ||}
 \hline
  Algorithm & $\pi(z)$ Before &  $\pi(z)$ After & Time (sec) & k\\
  \hline
  \hline
  Naive  & $0.35857$ & 0.23116 &  0.71483&5 \\
  \hline
  Merge & $0.35857$ & 0.23979 &  0.04088&5 \\
  \hline
  Distance &  $0.35857$ & 0.31491 &  0.00237 &5\\
  \hline
  \hline
  Naive &  $0.35857$ & 0.19331 &  0.54732 &10\\
  \hline
  Merge &  $0.35857$ & 0.18291 &  0.05307&10 \\
  \hline
  Distance &  $0.35857$ & 0.29321 &  0.00436&10\\
  \hline
  \hline
  Naive &  $0.35857$ & 0.17656 &  0.55274 &15\\
  \hline
  Merge &  $0.35857$ & 0.16946 &  0.04806&15\\
  \hline
  Distance &  $0.35857$ & 0.28158 &  0.00515&15\\
  \hline
  \hline
  Naive &  $0.35857$ & 0.14977 &  0.5511 &20\\
  \hline
  Merge &  $0.35857$ & 0.14569 &  0.05337&20\\
  \hline
  Distance &  $0.35857$ & 0.24387 &  0.00671&20\\
  \hline
 \end{tabular}
\end{table}

\begin{figure}[!htbp]
	\centering
	\includegraphics[width=0.65\textwidth]{Figures/karate_pol}
	\caption{Heuristic comparison of the decrease in Karate}
	\label{fig:karate_pol}
\end{figure}


\begin{figure}[!htbp]
	\centering
	\includegraphics[width=0.65\textwidth]{Figures/karate_time}
	\caption{Heuristic comparison of time in Karate}
	\label{fig:karate_time}
\end{figure}

\begin{table}[htbp]
 \centering
 \caption{Heuristics algorithms comparison on the Books}
 \label{tab:heuristicsKarate}
 \begin{tabular}{|c |l| c | c | c | c ||}
 \hline
  Algorithm & $\pi(z)$ Before &  $\pi(z)$ After & Time (sec) & k\\
  \hline
  \hline
  Naive  & $0.44046$ & 0.35138 &  28.76573&5 \\
  \hline
  Merge & $0.44046$ & 0.43515 &  0.3107&5 \\
  \hline
  Distance &  $0.44046$ & 0.43547 &  0.01219 &5\\
  \hline
  \hline
  Naive &  $0.44046$ & 0.30969 &  28.13354 &10\\
  \hline
  Merge &  $0.44046$ & 0.43006 &  0.32842&10 \\
  \hline
  Distance &  $0.44046$ & 0.41711 &  0.01954&10\\
  \hline
  \hline
  Naive &  $0.44046$ & 0.28730 &  26.88918 &15\\
  \hline
  Merge &  $0.44046$ & 0.39831 &  0.34483&15\\
  \hline
  Distance &  $0.44046$ & 0.39297 &  0.02893&15\\
  \hline
  \hline
  Naive &  $0.44046$ & 0.26987 &  29.96203 &20\\
  \hline
  Merge &  $0.44046$ & 0.36770 &  0.37486&20\\
  \hline
  Distance &  $0.44046$ & 0.38424 &  0.03249&20\\
  \hline
 \end{tabular}
\end{table}

\begin{figure}[!htbp]
	\centering
	\includegraphics[width=0.65\textwidth]{Figures/books_pol}
	\caption{Heuristic comparison of the decrease}
	\label{fig:books_pol}
\end{figure}


\begin{figure}[!htbp]
	\centering
	\includegraphics[width=0.65\textwidth]{Figures/books_time}
	\caption{Heuristic comparison of time}
	\label{fig:books_time}
\end{figure}



\section{Polarization in a complete graph}
\label{sec:fullgraph}
\vspace{20pt}
Given a polarized graph $G$ we will compute the polarization index $\pi(z)$ before and after converting the graph $G$ to a full graph. 

\begin{table}[!htbp]
 \centering
 \caption{Polarization Before and after converting to a full graph}
 \label{tab:fullgraph}
 \begin{tabular}{| l || l | l | l | l |}
 \hline
  Dataset & Number of Nodes & Number of edges & Average Degree & $\pi(z)$\\
  \hline
  \hline
  Karate Before & $34$ & $78$ & $4.5882$ &  $0.35857$\\
  \hline
  Karate After & $34$ & $561$ & $33$ &  $0.00081$\\
  \hline
  \hline
  Books Before & $105$ & $441$ & $8.4000$ &  $0.44046$\\
  \hline
  Books After & $105$ & $5460$ & $104.0000$ &  $0.00453$\\
  \hline
  \hline
  Blogs Before & $1490$ & $16718$ & $22.4403$ &  $0.27909$\\
  \hline
  Blogs After & $1490$ & $1109308$ & $1489.0040$ &  $0.00030$\\
  \hline
 \end{tabular}
 \end{table}

\vspace{20pt}
We can see the results from the karate, books and blogs datasets at table ~\ref{tab:fullgraph} The results leads us to the following lemma.
\\	
\begin{lemma}
The polarization index does not drop to zero in a fully connected graph.
\end{lemma}



\section{Experiment by removing edges}
\label{sec:properties}
Bellow we examine the removal of edges from a social graph and their result in polarization. We also use the edge betweenness centrality. The edge betweenness centrality is defined as the number of the shortest paths that go through an edge in a graph or network.(add cite Girvan and Newman 2002). 

In the tables following Sign and Addition refer to the multiplication and the addition of the opinions of the nodes that are attached to the specific edge examined. 
\\

\begin{table}[htbp]
 \centering
 \caption{Edges with the 5 largest decrease (Karate Dataset)}
 \label{tab:edgesLargest}
 \begin{tabular}{| l || l | l | l | l |}
 \hline
  Edge & Betweeness Centrality & Polarization Decrease & Sign & Addition\\
  \hline
  \hline
  (1, 32) & $0.12725$ & $0.04669$ & - &  0\\
  \hline
  (20, 34) & $0.059384$ & $0.03470$ & - &  0\\
  \hline
  (14, 34) & $0.06782$ & $0.02924$ & - &  0\\
  \hline
  (2, 31) & $0.03228$ & $0.02505$ & - &  0\\
  \hline
  (3, 28) & $0.04119$ & $0.02068$ & - &  0\\
  \hline
 \end{tabular}
  
 \caption{Edges with the 5 smallest decrease (Karate Dataset)}
 \label{tab:edgesLargest}
 \begin{tabular}{| l || l | l | l | l |}
 \hline
  Edge & Betweeness Centrality & Polarization Decrease & Sign & Addition\\
  \hline
  \hline
  (6, 7) & $0.00297$ & $0.0$ & + &  -2\\
  \hline
  (5, 11) & $0.00297$ & $5.55111*10^{-17}$ & + &  -2\\
  \hline
  (4, 8) & $0.00336$ & $3.04869*10^{-7}$ & + &  -2\\
  \hline
  (1, 4) & $0.02049$ & $1.38023*10^{-5}$ & + &  -2\\
  \hline
  (32, 34) & $0.05339$ & $1.61826*10^{-5}$ & - &  +2\\
  \hline
  \hline
 \end{tabular}
 
\end{table}

\begin{table}[htbp]
 \centering
 \caption{Edges with the 5 largest decrease (Blogs Dataset)}
 \label{tab:edgesLargest}
 \begin{tabular}{| l || l | l | l | l |}
 \hline
  Edge & Betweenness Centrality & Polarization Decrease & Sign & Addition\\
  \hline
  \hline
  (213, 793) & $0.00219$ & $0.00091$ & - &  0\\
  \hline
  (600, 1183) & $0.00439$ & $0.00074$ & - &  0\\
  \hline
  (523, 1375) & $0.00110$ & $0.00070$ & - &  0\\
  \hline
  (325, 1159) & $0.00110$ & $0.00069$ & - &  0\\
  \hline
  (632, 1000) & $0.00110$ & $0.00069$ & - &  0\\
  \hline
 \end{tabular}
 
 
 \caption{Edges with the 5 smallest decrease (Blogs Dataset)}
 \label{tab:edgesLargest}
 \begin{tabular}{| l || l | l | l | l |}
 \hline
  Edge & Betweenness Centrality & Polarization Decrease & Sign & Addition\\
  \hline
  \hline
  (384, 385) & $9.01465*10^{-7}$ & $0.0$ & + &  -2\\
  \hline
  (301, 644) & $2.11840*10^{-5}$ & $4.64017*10^{-13}$ & + &  -2\\
  \hline
  (775, 1369) & $4.23796*10^{-5}$ & $8.28614*10^{-13}$ & + &  -2\\
  \hline
  (233, 736) & $1.37432*10^{-5}$ & $1.26054*10^{-12}$ & + &  -2\\
  \hline
  (1330, 1410) & $4.08651*10^{-5}$ & $2.12324*10^{-12}$ & + &  +2\\
  \hline
  \hline
 \end{tabular}
 
\end{table}



\begin{table}[htbp]
 \centering
 \caption{Edges with the 5 largest decrease (Books Dataset)}
 \label{tab:edgesLargest}
 \begin{tabular}{| l || l | l | l | l |}
 \hline
  Edge & Betweeness Centrality & Polarization Decrease & Sign & Addition\\
  \hline
  \hline
  (53, 76) & $0.06290$ & $0.01985$ & - &  0\\
  \hline
  (46, 102) & $0.04914$ & $0.01541$ & + &  -2\\
  \hline
  (19, 77) & $0.04367$ & $0.01458$ & + &  +2\\
  \hline
  (9, 51) & $0.02812$ & $0.01000$ & - &  0\\
  \hline
  (49, 72) & $0.06809$ & $0.00952$ & - &  0\\
  \hline
 \end{tabular}
 
 
 \caption{Edges with the 5 smallest decrease (Books Dataset)}
 \label{tab:edgesLargest}
 \begin{tabular}{| l || l | l | l | l |}
 \hline
  Edge & Betweeness Centrality & Polarization Decrease & Sign & Addition\\
  \hline
  \hline
  (13, 40) & $0.00305$ & $3.89807*10^{-9}$ & + &  +2\\
  \hline
  (35, 37) & $0.00078$ & $8.65544*10^{-9}$ & + &  +2\\
  \hline
  (88, 89) & $0.00036$ & $1.00835*10^{-8}$ & + &  -2\\
  \hline
  (65, 69) & $0.00072$ & $1.51261*10^{-8}$ & + &  -2\\
  \hline
  (35, 36) & $0.00146$ & $2.92727*10^{-8}$ & + &  +2\\
  \hline
  \hline
 \end{tabular}
 
\end{table}


We can clearly see that there is not a direct association between the edge betweenness centrality and the decrease in polarization. For example in the karate dataset edge (20,34) has almost the same betweenness centrality with edge (32,34). The first is among the edges that their removal contributes in one of the biggest polarization decreases while the other is among the ones with the smallest. A second thing that we can see in all three datasets is that the biggest decrease is coming from the removal of edges that connect opposing opinions.



% Εισαγωγή της βιβλιογραφίας
\addstarredchapterc{\bibname} % minitoc
\bibliographystyle{IEEEtran}
\bibliography{/Users/leonidas/Desktop/February 21/thesis/LaTeXthesis/Template/Content/Bibliography}

% Προαιρετικά, μπορείτε να εισάγετε παραρτήματα
%\appendix 
%\include{Content/AppendixA}
%\include{Content/AppendixB}
%\include{Content/AppendixC}


% Εκτύπωση του ευρετηρίου (προαιρετικό)
%\printindex


% Σελίδες χωρίς αρίθμηση
\pagenumbering{gobble}

%\include{BackMatter/AuthorsPublications} % Προαιρετικό

% Σύντομο Βιογραφικό
%\include{BackMatter/ShortBiography}

\end{document}
