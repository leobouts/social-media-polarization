\chapter*{Περίληψη}
\addstarredchapter{Περίληψη} % minitoc

\noindent Γνωρίζουμε ως γεγονός το ότι οι γνώμες διαμορφώνονται μέσω των κοινωνικών συναναστροφών. Οι διαδικτυακές κοινότητες προσφέρουν δημόσια την πρόσβαση σε συζητήσεις για αμφιλεγόμενα ζητήματα που επιτρέπουν την μελέτη αλλά και τον έλεγχο τους. Η πλειοψηφία των ερευνών για τα κοινωνικά δίκτυα βασίζονται στο μοντέλο του Friedkin-Johnsen.
\\
\\
Οι χρήστες των διαδικτυακών κοινοτήτων λαμβάνουν μεροληπτικές πληροφορίες που ενισχύουν την δική τους οπτική. Αυτό δημιουργεί μία κατακερματισμένη κοινότητα και οι χρήστες αλληλεπιδρούν μόνο με άτομα που έχουν τις ίδιες γνώμες με αυτούς. Σε αυτήν την διπλωματική εργασία, θα χρησιμοποιήσουμε τον δείκτη πόλωσης για να μετρήσουμε το πόσο πολωμένο είναι ένα κοινωνικό γράφημα.
\\
\\
Προσπαθούμε να μειώσουμε την πόλωση με το να συνδέσουμε τα άτομα μεταξύ τους. Προτείνουμε νέες κοινωνικές συνδέσεις μεταξύ ατόμων που έχουν διαφορετικές και ακραίες γνώμες ακολουθώντας τον τρόπο που λειτουργεί το μοντέλο του Friedkin-Johnsen.
\\
\\
Τελικά, ενσωματώνουμε τις πιθανότητες στους ευριστικούς μας αλγορίθμους, που τώρα κάνουν επιλογές με βάση το πόσο πιθανό είναι να γίνει αποδεκτή μία πρόταση για μια νέα κοινωνική σύνδεση.

\bigskip
