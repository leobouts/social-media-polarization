\chapter*{Περίληψη}
\addstarredchapter{Περίληψη} % minitoc

Είναι γεγονός ότι οι γνώμες διαμορφώνονται μέσω των κοινωνικών συναναστροφών. Οι διαδικτυακές κοινότητες προσφέρουν δημόσια πρόσβαση σε συζητήσεις για αμφιλεγόμενα ζητήματα, επιτρέποντας την μελέτη αλλά και τον έλεγχο τους. Οι χρήστες των διαδικτυακών κοινοτήτων λαμβάνουν μεροληπτικές πληροφορίες που ενισχύουν την οπτική τους. Αυτό δημιουργεί μία κατακερματισμένη κοινότητα και οι χρήστες αλληλεπιδρούν μόνο με άτομα που έχουν τις ίδιες γνώμες με αυτούς. Σε αυτήν την διπλωματική εργασία θα χρησιμοποιήσουμε μια μετρική που προτείνεται στο \cite{tsapMatakosTerzi}, για να μετρήσουμε το πόσο πολωμένο είναι ένα κοινωνικό γράφημα, σύμφωνα με το δημοφιλές μοντέλο των Friedkin και Johnsen.
\\
\\
Προσπαθούμε να μειώσουμε την πόλωση με το να συνδέσουμε τα άτομα μεταξύ τους. Προτείνουμε νέες κοινωνικές συνδέσεις μεταξύ ατόμων που έχουν διαφορετικές και ακραίες γνώμες ακολουθώντας τον τρόπο που λειτουργεί το μοντέλο των Friedkin και Johnsen. Αρχικά, προσαρμόζουμε τους αλγορίθμους μας με σκοπό να ενσωματωσουμε την πιθανότητα αποδοχής στην επιλογή τους. Στη συνέχεια πραγματοποιουμε πειραματα με 6 διαφορετικα σετ δεδομένων. Παρατηρούμε μείωση στη πόλωση όταν συνδέουμε χρήστες με αντιτιθέμενες απόψεις. Όσο τα σετ δεδομένων γίνονται μεγαλύτερα, χρειάζεται να αυξήσουμε τον αριθμό των ακμών που προσθέτουμε, ώστε να διακρίνουμε σημαντική μείωση. Όταν ενσωματώνουμε πιθανότητες παρατηρούμε ότι η μείωση δεν είναι τόσο μεγάλη όσο τα προηγούμενα πειράματα. Αυτό οφείλεται στον συμβιβασμό που θα πρέπει να κάνουμε μεταξύ της προσθήκης των καλύτερων υποψηφίων, ώστε να μειώσουμε τη πόλωση χωρίς να γνωρίζουμε εάν θα γίνουν αποδεκτοί, και της επιλογής ακμών που πιθανότατα θα γίνουν αποδεκτοί, αλλά δεν έχουν τη μεγαλύτερη επίδραση στη μείωση της πόλωσης.

\bigskip
