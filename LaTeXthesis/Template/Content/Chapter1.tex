\chapter{Introduction and the Theory Behind Polarization}
\label{ch:Introduction}


\section{Introduction}
\label{sec:Objectives}

Polarization describes the division of people into two contrasting groups or sets of opinions or beliefs. The term is used in various domains such as politics and social studies. For example political polarization refers to the divergence of political attitudes to ideological extremes. Social studies use this term to  describe the segregation within a society in terms of income inequality or social and class status. Currently social media have a big role as a source of news and information and a lot of the related discussions of people have gone online. Polarization is linked with harmful effects such as intensifying stereotypes and creating echo chambers. In echo chambers individuals get their news only from like-minded people as they share and reinforce one another’s opinions. Additionally the fact that people tend to ignore opposing views in combination with algorithmic personalization results a significant increase of polarization.

\section{Social and Psychological Factors}

Individuals experience discomfort when given data that actively challenge their opinions. In the field of psychology, cognitive dissonance occurs when a person holds two or more contradictory beliefs, ideas, or values and experiences psychological stress because of that. In simple terms dissonance is defined as a the lack of agreement.
Individuals want to reduce the discomfort that is caused from cognitive dissonance. Reduction occurs by strengthening opinions that come in agreement with their own and downplaying everything that challenges them. This leads individuals to a selective exposure on information \cite{jonasHardtFreyThelen2001}. Selective exposure is also demonstrated in groups. Furthermore people assign themselves with social identities. The self-categorization theory stems from the social identity theory, which holds that conformity stems from psychological processes. Accordingly, proponents of the self-categorization model hold that group polarization occurs because individuals identify with a particular group and conform to a prototypical group position that is more extreme than the group mean. It is shown that groups of people tend to make decisions that are more extreme than the initial inclination of its members \cite{sunstein}.

\label{sec:Structure}

\section{Polarization online}

Online entities such as news or social media platforms are aware of their users opinions  and aim to maximize their satisfaction. As discussed above, platforms will present content in a way that minimizes psychological stress. This leads to media bias. Media bias is the bias or perceived bias of journalists and news producers within the mass media in the selection of many events and stories that are reported and how they are covered. When this happens online, personalization of the content creates algorithmic bias. Algorithmic bias describes systematic and repeatable errors in a computer system that create unfair outcomes, such as privileging one arbitrary group of users over others. Bias can emerge due to many factors like the design of the algorithm. Due to personalization we don't see the same content and this is the main reason for the formation of filter bubbles.

\section{Filter Bubbles}

Filter bubbles are the echo chambers of social media. In news media, an echo chamber is a metaphorical description of a situation in which beliefs  and opinions are strengthened by communication and repetition inside a closed system. It is important to distinguish the difference between echo chambers and filter bubbles. This two concepts are almost identical, however,  filter bubbles are a result of algorithms that choose content based on previous online behaviour, as with search histories or online shopping activity.


