\chapter{Introduction}
\label{ch:Introduction}


\section{Motivation and Thesis Goal}
\label{sec:motivation}

Real world events such as Brexit and the 2016 U.S. presidential elections give us a clear hint about the polarization our society is witnessing. Social media polarization has a strong effect on politics, opinion formation and how people interact with each other in a society. In a polarized social network, users receive biased information that amplifies their own viewpoints. 
\\
\\
Polarization describes the division of people into two contrasting groups or sets of opinions or beliefs. The term is used in various domains such as politics and social studies.
In social media, users tend to join communities of like-minded individuals and the opinions of the users are amplified and reinforced by the continuous  communication and recycling of the same views. These communities are referred to as echo chambers.
Echo chambers can be created where information is exchanged, whether it is online or in real life. 
On social media almost anyone can quickly find like-minded people and countless news sources. 
This has made echo chambers far more numerous and easy to fall into.
\\
\\
An echo chamber leads its members to distrust everybody on the outside of that chamber. This can lead to countless problems on politics, public discourse and poses a threat to the way democracies work. To shield our societies, there is a need for tools for reducing polarization.


\section{Thesis Contributions}
\label{sec:outline}

In this thesis, we will consider the problem of reducing polarization by proposing new social connections. For measuring polarization we consider the polarization index measure introduced in~\cite{tsapMatakosTerzi}. This metric is based on the popular Friedkin and Johnsen model. This model assumes that users have an internal and an expressed opinion, and that the expressed opinion of a node is computed through repeated averaging of her internal opinion and the expressed opinions of their social circle. The polarization index is defined as the measure of the vector of the expressed opinions. The idea is that the closer this measure to zero, the closer the network to neutrality. We then proceed and define two problems. We ask for the best $k$ edges that, if introduced in the network, they will lead to the greatest reduction of the polarization. 
\\
\\
The second problem takes into account that in a real social network, new social connections are not always accepted. For example, we would not accept friend requests from people we barely know. We thus assume that every missing edge has some probability to appear. These probabilities can be estimated using link recommendation algorithms. The second problem we consider, $k-expected-addition$, asks for the best $k$ edges that if introduced in the network, they will lead to the greatest \emph{expected} reduction of the polarization index.
\\
\\
Our heuristics are based on the intuition that the Friedkin and Johnsen model has the biggest polarization decrease when we connect different and extreme opinions. We classify our heuristics in two categories. In these two categories, the heuristics do or do not recompute the opinion vector after the addition of an edge. This is derived from the fact that when adding an edge to the network the structure of the graph changes. The heuristics that consider network changes are the $Greedy$ the $FTGreedy$ and the $Expressed Opinion$. These three are then modified into a batch version that does not consider network changes. We continue by using Graph Embeddings and the $Node2Vec$ algorithm to compute acceptance probabilities. We use these probabilities in a modified version of our heuristics to compute how much we expect the polarization metric to drop. Our heuristics are applied in 6 datasets of various topics and compared with each other. The Greedy heuristics cannot run on graphs that contain a lot of nodes due to time limitations.

\section{Roadmap}
\label{sec:roadmap}

Chapter~\ref{ch:related} addresses the related work around polarization and decreasing polarization. We see how polarization is measured, the relation between polarization and random walks and how polarization can be combined with disagreement and conflict. We also briefly review the work on  the $Node2Vec$ algorithm. Chapter~\ref{ch:premAndDef} defines the Friedkin and Johnsen model and the polarization metric we use. Our two problems are also defined there, the $k-Addition$ problem and the $k-Addition-Expected$. Then, a counter-example is provided for the monotonicity of the polarization metric we use. In Chapter~\ref{ch:algorithms},  we describe our algorithms for both problems. Chapter~\ref{ch:experiments} presents our experiments for the two problems we consider.




