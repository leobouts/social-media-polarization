\chapter{Introduction and the Theory Behind Polarization}
\label{ch:Introduction}


\section{Introduction}
\label{sec:Objectives}

Polarization describes the division of people into two contrasting groups or sets of opinions or beliefs. The term is used in various domains such as politics and social studies. For example political polarization refers to the divergence of political attitudes to ideological extremes. Social studies use this term to  describe the segregation within a society in terms of income inequality or social and class status.
\\
\\
Currently social media have a big role as a source of news and information and a lot of the related discussions of people have gone online. Polarization is linked with harmful effects such as intensifying stereotypes and creating echo chambers. 
\\
\\
In echo chambers individuals get their news only from like-minded people as they share and reinforce one another’s opinions. Additionally the fact that people tend to ignore opposing views in combination with algorithmic personalization results a significant increase of polarization.

\section{Social and Psychological Factors}

Individuals experience discomfort when given data that actively challenge their opinions. In the field of psychology, cognitive dissonance occurs when a person holds two or more contradictory beliefs, ideas, or values and experiences psychological stress because of that. In simple terms dissonance is defined as a the lack of agreement.
\\
\\
Individuals want to reduce the discomfort that is caused from cognitive dissonance. Reduction occurs by strengthening opinions that come in agreement with their own and downplaying everything that challenges them. This leads individuals to a selective exposure on information \cite{jonasHardtFreyThelen2001}. Selective exposure is also demonstrated in groups. Furthermore people assign themselves with social identities. 
\\
\\
The self-categorization theory stems from the social identity theory, which holds that conformity stems from psychological processes. Accordingly, proponents of the self-categorization model hold that group polarization occurs because individuals identify with a particular group and conform to a prototypical group position that is more extreme than the group mean. It is shown that groups of people tend to make decisions that are more extreme than the initial inclination of its members \cite{sunstein}.

\label{sec:Structure}

\section{Polarization online}

Online entities such as news or social media platforms are aware of their users opinions  and aim to maximize their satisfaction. As discussed above, platforms will present content in a way that minimizes psychological stress. This leads to media bias. 
\\
\\
Media bias is the bias or perceived bias of journalists and news producers within the mass media in the selection of many events and stories that are reported and how they are covered. When this happens online, personalization of the content creates algorithmic bias. 
\\
\\
Algorithmic bias describes systematic and repeatable errors in a computer system that create unfair outcomes, such as privileging one arbitrary group of users over others. 
\\
\\
Bias can emerge due to many factors like the design of the algorithm. Due to personalization we don't see the same content and this is the main reason for the formation of filter bubbles.

\section{Filter Bubbles}

Filter bubbles are the echo chambers of social media. In news media, an echo chamber is a metaphorical description of a situation in which beliefs  and opinions are strengthened by communication and repetition inside a closed system. It is important to distinguish the difference between echo chambers and filter bubbles. This two concepts are almost identical, however,  filter bubbles are a result of algorithms that choose content based on previous online behaviour, as with search histories or online shopping activity.


\section{Polarization and Society}

Political polarization can be defined as the difference in ideological extremes but in political science almost in every context polarization is considered as the gap between the political parties of a society.
\\
\\
Most of the time political parties disagree on policy issues and that is the main drive of democracy. With heightened polarization the followers of each political party start fearing that the other will destroy their society with their agendas. Destroying the other side becomes their only objective and this is how democracies fall apart.
\\
\\
Social networks are frequently liable for terrorism. Terrorist leaders create communities of individuals that have the same opinions and fuel them with each other. As mentioned in ~\ref{sec:Structure}  when like-minded people discuss with each other they tend to move toward extreme positions. This has a bigger effect when people are already quite extreme. 
\\
\\
Terrorist leaders know this and they try to make sure that all individuals inside this community will speak and interact with people that have the same extreme direction. If members of the community think that they have a shared identity the polarization will grow. Terrorist leaders will also repress opposing views and will not tolerate internal disagreement. They take every step needed to ensure unity.
\\
\\
Most individuals lack in confidence on their own views or have more confidence that are willing to show. Fear of marginalization or being proven wrong make them present a moderate version of theirselves. In either case, group dynamics can push people toward a more extreme position.
\\
\\
Social influence also plays a great role. People have a certain image of themselves and how they want to be viewed by others. Most people like to think of themselves as not identical to but as different from others, but only in the right direction and to the right extent. There is evidence that social influence is an independent factor behind group polarization; consider in particular the fact that mere exposure to the views of others can have this effect, even without any discussion at all \cite{sunstein}.
\\
\\
Combining these factors we obtain a highly dangerous and highly polarized community for everyday life.


