\chapter{Related Work}
\label{ch:Instructions}


\section{Measuring the Polarization of a Network}
\label{sec:Submission}

At first we have to measure the opinion polarization in a social Network. The actions and information of a user can give us insights about his opinions on a topic e.g. accounts a user follows, content they repost, comments they make , etc. Using this information we can measure the polarization. 
\\
\\
Assume a graph $G = (V,E)$ representing a network that is connected and undirected. $Z$ will be the vector of expressed opinions for the whole network. Each value $Z_i$ of the vector will represent a node and can be computed with the opinion-formation model of Friedkin and Johnsen. 
\\
\\
The length of the opinion vector $||z|| ^2$ measures  the polarization and  $\pi(z) = \frac{||z|| ^2}n$ is defined as the polarization index of the network, where  $n$ is the number of nodes in the graph so the polarization index can be independent of the network size. 
\\
\\
There is a direct link between this opinion model and random walks. Given the graph $G = (V,E)$ we can construct the augmented graph $H(V∪X, E∪R)$. For each vertex of  $V$ we will add a new vertex on $X$ and a directed edge $(v_i,x_i)$ in $R$. 
\\
\\
The node $x_i$ corresponds to the internal opinion of the node $v_i$. In the model we follow $z_j$ or else the expressed opinion of a user that can be computed by the probability of $P(x_i |v_j)$. This probability represents that a random walk on the augmented graph that started from the node $V_j$ ended at the node $X_i$ or else how much likely the probability of user $V_j$ adopting the opinion of user $V_i$. This probability depends on the structure of the graph. 
\\
\\
Two problems are introduced, the $ModerateInternal$ and the $ModerateExpressed$. When moderating opinions a small set of nodes $T_s$ is being set to zero, in each problem, as their names suggests, internal or external opinions are set to zero. Two algorithms are proposed for the $ModerateInternal problem$. 
\\
\\
A greedy algorithm that finds the set $T_s$ of nodes iteratively according to the biggest decrease it causes and the Binary Orthogonal Matching Pursuit (BOMP) algorithm. For the $ModerateExpressed$ problem the same greedy algorithm is used.
\cite{tsapMatakosTerzi}


\section{Polarization and Disagreement}
\label{sec:polarizationDisareement}

Another way of looking at polarization is by combining it with disagreement. The main problem of minimising polarization and disagreement lies in the opinions of each user and how targeted ads and recommendations influence their opinions. 
\\
\\
Considering the disagreement in combination with polarization a network can choose how to respond in different situations. Their recommendation system could choose between keeping the disagreement low or exposing users to radically different opinions. There are situations that this optimisation can reduce the overall polarization-disagreement in the network by recommending edges in different parts of the network than the ones that agree with the human confirmation bias. 
\\
\\
Given a social network $G = (V,E,w)$ and initial opinions $s: V \rightarrow [0,1]$ the equilibrium vector according to the Friedkin-Johnsen model is defined as $z^*=(I+L)^1s$ where $L$ is the laplacian matrix of the graph and $I$ the identity matrix. Disagreement of $d(u,v)$ of edge $(u,v)$ is defined as the squared difference between the opinions of $u,v$ at equilibrium: $d(u,v) = w_{uv}(Z_u^* - Z_v*)^2.$ 
\\
\\
The total disagreement is defined as $D_{G,s} = \sum_{(u,v) \epsilon E} d(u,v)$. With $\bar z = z^* - \frac{z*^T \overrightarrow 1}{n}\overrightarrow 1$ polarization is measured as a deviation from the average with the standard definition of variance as $P_{G,s} = \sum_{u\epsilon V} \bar z_u^2 = \bar z^T \bar z$ .
\\
\\
The polarization-disagreement index is defined as follows $I_{G,s} = P_{G,s} +D_{G,s}$. The objective is to minimize this index. 
\\
\\
Muco and Tsourakakis have shown that minimising $\bar z^T \bar z + \bar z^T L\bar z$ is the same to minimising the polarization-disagreement index. Here, $L$ is a matrix among the set of valid combinatorial Laplacians of connected graphs.\cite{musco}

\section{Quantifying and Minimizing Risk of Conflict in Social Networks}
\label{sec:riskOfConflict}

We know for a fact that opinions are formed through social interactions and in every interaction conflict arises. Online networks offer public access to social disputes on controversial matters that allows the study and moderation of them. The majority of studies are based in the Friedkin-Johnsen model. 
\\
\\
The main problem is with the Friedkin-Johnsen model metrics. The external opinion of a user, which by definition is hard to measure, combined with the internal opinion which is impossible to be measured. Another problem occurs in the editing of the social graph. We edit the social graph in a way that minimises the conflict of a certain social issue. This can lead to an increased conflict of one or more social issues inside the network.
\\
\\
\\
Chen, Lijffijt and De Bie still use the Friedkin-Johnsen model to evaluate the network conflict but the quantifications depend only on the network topology in a way that the conflict can be reduced over all issues. Worst-case(WCR) conflict risk and average-case conflict risk(ACR) are defined to represent two separate problems, how the network can be minimised in the worst case or in the average case scenario by altering the social graph. 
\\
\\
These problems consider the measures of internal conflict, external conflict, and controversy. Internal conflict ($ic$) measures the difference of the internal and the expressed opinion of a user. $ic = \sum_i{(z_i-s_i)^2}.$
\\
\\
 External conflict ($ec$) measures how different are the opinions of the neighbours with each other. $ec = \sum_{(i,j) \epsilon E}{w_{ij}(z_i-z_j)^2}$. 
 \\
 \\
 Controversy ($c$) measures the variation of the opinions in the network and is independent of the social graph structure. $c = \sum_i{z_i}^2.$
 \\ 
 \\
 These measures are not independent. Reducing one of them results in the increase of another. This leads to the conservation law of conflict. $S^TS = ic + 2ec +  c.$ 
 \\
 \\
 There are two methods of minimising the conflict of the network for each of the ACR and WCR problems. One is a gradient method that  considers deleting and adding edges simultaneously and the other is a descent method that suggests deleting or adding a single edge. Chen, Lijffijt and De Bie used small world random networks and random networks with binomial and power law degree distribution to find out what types of networks have the highest risks for every conflict measure they defined. 
 \\
 \\
 A small world network is a type of graph in which most nodes are not neighbours of one another, but the neighbours of any given node are likely to be neighbours of each other and most nodes can be reached from every other node by a small number of hops or steps. They found that the small world networks are the most high-risk for the $ic$ metric. For $c$ and $r$ the most high-risk network depends on the density.\cite{chen}

\section{Reducing Controversy by Connecting Opposing Views}
\label{sec:reducing}

Garimella et al. relly on a measure of controversy that is shown to work reliably in multiple domains in contrast with other measures that focus on a single topic. The controversy measure consists of the following steps:

\begin{enumerate}
  \item Given a topic $t$ they create an endorsement gragp $G=(V,E)$. This graph represents users who have generated content relevant to $t$. For example hashtags of a user.
  
  \item The nodes of this graph a re partitioned in two disjoint sets $X$ and $Y$. The partition is obtained using a graph-partition algorithm.
  
  \item The last step, is computing the controversy measure through a random-walk, thus creating the controversy score $RWC$. This score is defined as the difference of the probability that a random walk starting on one side of the partition will stay on the same side and the probability that the random walk will cross to the other side. A personalized PageRank is used where the restart probabilities are set to a random vertex of each side.
\end{enumerate}
\vspace{4pt}
Garimella et al. states that real graphs often have a star-like structure. Small number of higly popular vertices have a lot of incoming edges. These nodes can be seen as thought leaders and their followers. It is shown that connecting the high degree vertices minimises the $RWC$ score.
\\
\\
Probabilities are also incorporated in the sense that a new edge addition may be not accepted by the user. The polarity here is defined as $R_u= p^X(u) - p^Y(u) \epsilon [-1,1]$. 
\\
\\
The definition of $p^X(u)$ and $p^Y(u)$ is the fraction of other vertices $u'$ for which $lu'^X<lu^X$ and $lu'^Y<lu^Y$. 
\\
\\
In addition $lu^X$ and $lu^Y$ stand for the expected time a random walk needs to hit the high degree vertices of $X$ and $Y$ respectively starting from u. Considering $u$ and $v$ as 2 different and not connected users $P(u, v)$ is defined as the probability that $u$ accepts a recommendation to connect with $v$.
 \\
 \\
 Let $R_u$ and $R_v$ the polarity of these users respectively. $P(u, v)$ is estimated from the training data by obtaining $N_{Endorsed(R_u,R_v)} / N_{Exposed(R_u,R_v)}$.
 \\
 \\
The $Endorsed(R_u,R_v)$ and $Exposed(R_u,R_v)$ values represent the number of times a user with polarity $R_v$ was exposed/endorsed content generated by a user with $R_u$. For example $v$ follows $u$, thus $v$ is exposed to all content $u$ generates.
\\
\\ 
Finally we can re-define the problem as the expected decrease of $RWC$.
$E(u,v) = p(u,v) * δRWC_{u \rightarrow v}$

