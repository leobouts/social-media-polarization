\chapter{Algorithms}
\label{ch:algorithms}



\section{Intuition}
\label{sec:intuition}

To solve this problem we have to evaluate all possible edge combinations. Even for greedy heuristics we need to limit the edge candidates. The algorithm considers nodes with a high expressed value. According to our model the smallest decrease is happening when we connect a value near zero and a relatively high value. Consider an arbitraty example with two nodes. node a has z_a of -0.02 and node b has z_b of 0.5. the weights are = 1. both of the z_i denominators will be increased by one. The numerator of the one node a will be increased by a lot and the numerator of the node b will be decreased by a a small value. The new z_a will not change a lot because the big addition in the numerator will approach the +1 addition of the denominator. On the other hand the new z_b will see a big change as the numerator had a small decrease thus creating a big decrease for this node. We can clearly see that only one of the two nodes will come closer to a big decrease. 

Now consider a second example of two nodes node c = -0.8 and node d = 0.9. After the addition node d will have a big decrease because we add two conflicting values that almost neutralize each other but the addition of the +1 on the denominator stands still. On the other hand node C will also see a big decrease for the same reason. With this type of connection both of the nodes have a significant decrease.


### fix the above to proper english sentences u fucking donkey


## remove previous stuff from bellow, adjust the algorithm table


\\
\\
Let $G$ be our social network graph. In the figure~\ref{fig:starA} bellow we can see how it resemble a star like shaped network. Nodes $0-4$ have an internal value of $Z_i = 1$ and nodes $5-9$ have an internal value of $Z_i=-1$. Now we are going to compute the polarization index for the network $G$. 
\\



Therefore, the maximum decrease is achieved by the addition of node $4\rightarrow6$. Even though real graphs do not match this case exactly, they often have a structure that resembles star-graphs in certain ways: a small number of highly popular vertices receive incoming edges from a large number of other vertices \cite{garimella}.

\section{Proposed Algorithm}
\label{sec:proposedAlgorithm}

The results from section~\ref{sec:intuition} makes us consider edge addition between low-degree vertices from each side of the polarized communities. The algorithm can be seen in the figure Algorithm~\ref{alg:algorithm}.

\begin{algorithm}[t]
	\caption{Minimization of the polarization index $\pi(z)$}
	\label{alg:algorithm}
	\begin{flushleft}
        		\textbf{INPUT:} Graph $G$, number of edges to add , $k$; $k1$, $k2$, High polarization vertices of each opinion , [-1,0] and [0,1], $X$,$Y$ respectively.\\
        		\textbf{OUTPUT:} List of $k$ edges that minimize the polarization index $\pi(z)$
	\end{flushleft}
	\begin{algorithmic}[1]
		\STATE $Out \leftarrow Empty List;$
		\FOR {$i = 1:k1 \ do$}
		\STATE $Vertex \ u = X[i]$
		\FOR {$j= 1:k2 \ do$}
		\STATE $Vertex \ v = Y[i]$
		\STATE Compute $\pi(z)$, the decrease if the edge $(u,v)$ is added;
		\STATE Append edge $(u,v)$ to Out;
		\ENDFOR
		\ENDFOR
		\STATE $Sorted \leftarrow sort(Out)$ by $\pi(z)$ by decreasing order;
		\STATE Return top $k$ from $Sorted$
	\end{algorithmic}
\end{algorithm}


