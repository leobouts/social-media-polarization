\chapter{Algorithms}
\label{ch:algorithms}


In this section we  consider a greedy algorithm and some heuristics for reducing $\pi(z)$. All the heuristics use the intuition that connecting the most extreme opinions of each community can result in great reduction. 
When a new edge is introduced, the graph structure changes. This leads to changes in the opinion vector $z$.
The recomputation of the $z$ vector is expensive on time due to the computation of the inverse matrix in the $(L+I)^{-1}S$ formula.
This is why we consider two types of algorithms, those that recompute the $z$ vectors and those that do not.
All the heuristics run on $\mathcal{O}(n^2)$. This is due to the fact that they need to explore all edge combinations.

\section{Algorithms that recompute the opinion vector}
\label{sec:recomputeAlgos}

We begin with a Greedy algorithm. Greedy algorithms work in stages and during each stage a choice is made which is locally optimal.
The Greedy algorithm computes the decrease in $\pi(z)$ and selects the edge with the largest decrease every time.
\\
\\
After finding the best edge, the $Greedy$ algorithm adds this edge to the graph. This result in a change of the network structure.
Then a recomputation of the $z$ vector is happening and  the procedure is repeated.To reduce running times, we  use repeated averaging instead of computing the inverse matrix and limit the accuracy on the convergence.

\vspace{30pt}
    		\begin{algorithm}[H]
		
			\caption{Greedy}
			\label{alg:greedyAlgo}
			
			\begin{flushleft}
        				\textbf{INPUT:} Graph $G(V, E)$; $k$ number of edges to add;
				\vspace{6pt} \\
        				\textbf{OUTPUT:} A set S of k edges to be added to $G$ that minimize the polarization \\
				 index $\pi(z)$
			\end{flushleft}
			
			\begin{algorithmic}[1]
				\FOR {$i = 1:k \ $}
					\FOR { each  edge in $|V| \times |V| \textbackslash E$}
						\STATE Compute the decrease of $\pi(z)$ if edge is added to $G$;
					\ENDFOR
					\STATE Select the edge with the largest decrease and add it to $G$.
				\ENDFOR
				\STATE Return the set of edges that were selected.
			\end{algorithmic}
		\end{algorithm}
\vspace{30pt}

\noindent A second heuristic we consider is the $FirstTopGreedy$. Let $X$ be the set of nodes of expressed opinions $\epsilon$ [-1,0) sorted by increasing order and $Y$ the set of nodes of expressed opinions $\epsilon$ (0,1] sorted by decreasing order. This heuristic use the first $k$ nodes of $X$ and $Y$, resulting in  a $k \times k$ smaller search space. This allows the $FirstTopGreedy$ to reduce the amount of time spend searching for the best edge to add. \noindent 
\\
\\
Last we consider two heuristics that choose edges based on the value of the expressed opinion of their nodes. The Distance of their opinions can be defined as $D=|z_u - z_v|$. This heuristic computes the distance between every edge candidate and then chooses to add the edge with the maximum distance.

\begin{algorithm}[htbp]
	\caption{FirstTopGreedy}
	\label{alg:kgreedy}
	
	\begin{flushleft}
        		\textbf{INPUT:} Graph $G(V, E)$; $k$ number of edges to add;\\
		$X$, the set of nodes that their expressed opinions $\epsilon$ [-1,0) sorted by increasing order\\
		$Y$, set of nodes that their expressed opinions $\epsilon$ (0,1]  sorted by decreasing order\\
		\vspace{6pt}
        		\textbf{OUTPUT:} A set S of k edges to be added to $G$ that minimize the polarization \\ index $\pi(z)$
	\end{flushleft}
	
	\begin{algorithmic}[1]
		\STATE $A \leftarrow $ first $k$ items of $X$
		\STATE $B \leftarrow $ first $k$ items of $Y$
		\FOR {$i = 1:k \ $}
			\STATE$Decrease \leftarrow Empty List$;
			\FOR { each  edge in $|A| \times |B| \textbackslash E$}
				\STATE Compute the decrease of $\pi(z)$ if edge is added to the graph;
				\STATE Append the decrease on the $Decrease$ list;
			\ENDFOR
			\STATE Select the edge with the largest decrease from the $Decrease$ list.
			\STATE Add this edge to the graph.
		\ENDFOR
		\STATE Return the set of edges that were selected.
	\end{algorithmic}
	
\end{algorithm}
		
\clearpage


\begin{algorithm}[H]
	\caption{ExpressedΟpinion}
	\label{alg:expressedOpinion}
	
	\begin{flushleft}
        		\textbf{INPUT:} Graph $G(V, E)$; $k$ number of edges to add\\
		\vspace{6pt}
        		\textbf{OUTPUT:} A set S of k edges to be added to $G$ that minimize the polarization
		\\ index $\pi(z)$
	\end{flushleft}
	
	\begin{algorithmic}[1]
		\FOR {$i = 1:k \ $}
			\STATE $EdgesToAdd \leftarrow Empty List;$
			\STATE Compute the $z$ values.
			\FOR { each  edge in $|V| \times |V| \textbackslash E$}
				\STATE Append to $EddgesToAdd$ the value $D=|z_u - z_v|$.
			\ENDFOR
			\STATE $Sorted \leftarrow sort(EdgesToAdd)$ by decreasing order;
			\STATE Add the first edge of $EdgesToAdd$ to the graph
		\ENDFOR
		\STATE Return the set of edges that were selected.
	\end{algorithmic}
	
\end{algorithm}

\section{Algorithms that do not recompute the opinion vector}
\label{sec:norecomputeAlgo}

Bellow we explore some heuristics that do not consider the network changes and thus can run more easily in larger datasets.
Computing the $\pi(z)$ is an expensive operation due to the computation of the inverse matrix. We compute the opinion vector only once and we sort the edges according to the decrease. Then we select the top $k$ edges. At first we can see a variation of the $Greedy$ algorithm. Its implementation is similar to the $Greedy$ but without recomputing the  $\pi(z)$ at each round. 
\\
\\We continue by using a variation of the $FirstTopGreedy$.The $FirstTopGreedyBatch$ heuristic. $FirsTopGreedy$  works exactly like $GreedyBatch$ with the difference that the opinion vector is not recomputed. Let $X$ be the set of nodes that their expressed opinions $\epsilon$ [-1,0) sorted by increasing order and $Y$ the set of nodes that their expressed opinions $\epsilon$ (0,1] sorted by decreasing order. This heuristic is taking the first $k$ nodes of $X$ and $Y$, resulting in $k \times k$ nodes.
\clearpage

\begin{algorithm}[H]
		
			\caption{GreedyBatch}
			\label{alg:greedyBatch}
			
			\begin{flushleft}
        				\textbf{INPUT:} Graph $G(V, E)$; $k$ number of edges to add;
				\vspace{6pt}\\
        				\textbf{OUTPUT:} A set S of k edges to be added to $G$ that minimize the polarization \\ index $\pi(z)$
			\end{flushleft}
			
			\begin{algorithmic}[1]
				\STATE $EdgesToAdd \leftarrow Empty List;$
				\FOR { each  edge in $|V| \times |V| \textbackslash E$}
					\STATE Compute the decrease of $\pi(z)$ if edge is added to $G$ and add it to $edgesToAdd$
				\ENDFOR
				\STATE $Sorted \leftarrow sort(EdgesToAdd)$ by decreasing order;
				\STATE Select the $k$ edges with the largest decrease and add it to $G$.
				\STATE Return the set of edges that were selected.

			\end{algorithmic}
			
		\end{algorithm}


\begin{algorithm}[H]
	\caption{FirstTopGreedyBatch}
	\label{alg:kgreedy}
	
	\begin{flushleft}
        		\textbf{INPUT:} Graph $G(V, E)$; $k$ number of edges to add;\\
		$X$, the set of nodes that their expressed opinions $\epsilon$ [-1,0) sorted by increasing order\\
		$Y$, set of nodes that their expressed opinions $\epsilon$ (0,1]  sorted by decreasing order\\
		\vspace{6pt}
        		\textbf{OUTPUT:} A set S of k edges to be added to $G$ that minimize the polarization \\ index $\pi(z)$
	\end{flushleft}
	
	\begin{algorithmic}[1]
		\STATE $A, B \leftarrow $ first $k$ items of $X$ , $Y$
		\STATE$Decrease \leftarrow Empty List$;
		\FOR { each  edge in $|A| \times |B| \textbackslash E$}
			\STATE Compute the decrease of $\pi(z)$ if edge is added to the graph;
			\STATE Append the decrease on the $Decrease$ list;
		\ENDFOR
		\STATE Add the first $k$  edges from the $Decrease$ list and return them.
	\end{algorithmic}
	
\end{algorithm}
\clearpage

