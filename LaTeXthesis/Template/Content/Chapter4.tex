\chapter{Algorithms}
\label{ch:algorithms}



\section{Intuition}
\label{sec:intuition}

To solve this problem we have to evaluate all possible edge combinations. Even for greedy heuristics we need to limit the edge candidates. The algorithm considers nodes with a high expressed value. According to our model the smallest decrease is happening when we connect a value near zero and a relatively high value.
\\
\\
We will now see why this statement holds by examining how the expressed opinion changes with an addition in the Friedkin and Johnsen model. Consider an arbitrary example with two nodes inside a network. Node $a$ has $z_a = -0.02$ and node $b$ has $z_b = 0.5$. Also for this example we assume that $w_{ii}=w_{ij}=w_{ji}=1$. 
\\
\\
If we connect these two nodes with an edge and re calculate the expressed opinions both of the $z_i$ denominators will be increased by one. This emerges from the fact that both nodes will have an additional neighbour and that all weights equal with one. The numerator of the one node $a$ will be increased by a lot and the numerator of the node $b$ will be decreased by a small value.
\\
\\
The new $z_a$ will not change a lot because the big addition in the numerator will approach the +1 addition of the denominator. On the other hand the new $z_b$ will see a big change as the numerator had a small decrease thus creating a big decrease overall for this node. We can clearly see that only one of the two nodes will amount to a big decrease. 
\\
\\
Now consider a second example of two nodes node $c$ has $z_c = -0.8$ and node $d$ has $z_d = 0.9$. After the addition node $d$ will see a big decrease because we add two conflicting values that almost neutralise each other on the numerator but the addition of the +1 on the denominator stands still. On the other hand node $c$ will also see a big decrease for the same reason. With this type of connection both of the nodes have a significant decrease.
\\
\\
\section{Properties of nodes having a large decrease}
\label{sec:properties}

When analysing a social graph we need centrality measures. We will examine nodes that have a large decrease of the polarization index in contrast with nodes that the decrease is almost insignificant. One of the centrality measures is the betweenness centrality. This metric measures the number of times a node lies on the shortest path between other nodes and shows which nodes are bridges between nodes in a network. The other is the closeness centrality. This metric scores each node based on their closeness to all other nodes in the network by calculating the shortest paths between all nodes and assigning each node a score based on its sum of shortest paths.
\\
\\
The karate club social network\cite{nr} consists of a small pollarized community between two opposing karate teachers and their followers. After finding all possible connections between opposing opinions we compare the betweenness and closeness centrality of them.
\\
\\

\begin{table}[t]
 \centering
 \caption{Nodes with the largest decrease}
 \label{tab:nodesLargest}
 \begin{tabular}{| l || l | l | l |}
 \hline
  Node id & Betweeness Centrality& Closeness Centrality & Eigen Centrality\\
  \hline
  \hline
  6 & 0.38372 & 0.02998 & 0.07948\\
  \hline
  7 & 0.38372 & 0.02998 & 0.07948\\
  \hline
  17 & 0.28448 & 0.0 &  0.02363\\
  \hline
  24 & 0.39285 & 0.01761 & 0.15012\\
  \hline
  26 & 0.375 & 0.00384 & 0.05920\\
  \hline
  27 & 0.36263 & 0.0 & 0.07558\\ 
  \hline
  30 & 0.38372 & 0.00292 & 0.13496\\ 
  \hline
  \hline
  Norm & 0.97422 & 0.04617 & 0.251324\\ 
  \hline
 \end{tabular}
\end{table}

\begin{table}[t]
 \centering
 \caption{Nodes with the smallest decrease}
 \label{tab:nodesSmallest}
 \begin{tabular}{| l || l | l | l |}
 \hline
  Node id & Betweeness Centrality& Closeness Centrality & Eigen Centrality\\
  \hline
  \hline
  3 & 0.55932 & 0.14365 & 0.31718\\
  \hline
  9 & 0.51562 & 0.05592 & 0.22740\\
  \hline
  10 & 0.43421 & 0.00084 &  0.10267\\
  \hline
  14 & 0.51562 & 0.04586 & 0.22646\\
  \hline
  20 & 0.5 & 0.03247 & 0.147911\\
  \hline
  28 &  0.02233 & 0.0 & 0.13347\\ 
  \hline
  29 & 0.00179 & 0.00292 & 0.13107\\
   \hline
  31 & 0.01441 & 0.00292 & 0.17476\\ 
   \hline
  32 & 0.13827 & 0.00292 & 0.19103\\ 
   \hline
  33 & 0.14524 & 0.00292 & 0.30865\\ 
   \hline
  34 & 0.30407 & 0.00292 & 0.37337\\  
  \hline
  \hline
    Norm & 1.66392 & 0.40038 & 0.75679\\ 
  \hline
 \end{tabular}
\end{table}



