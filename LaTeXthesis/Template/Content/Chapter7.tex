\chapter{Conclusions}
\label{ch:conclusions}


At first we explored some heuristics to reduce the polarization. The $Greedy$ heuristic cannot run in large datasets. We tried to limit the search space by using the $FirstTopGreedy$ heuristic. Even in a smaller space larger datasets need a lot of time to compute the inverse matrix and can only run for a small $k$. The batch heuristics were used to save time by not recomputing the opinion vector $z$ but perform very poorly and in some cases even worse than the random algorithm. This is derived from the fact that when adding a new edge the opinion vector $z$ changes and without recomputation the batch algorithms will not choose a good candidate for reducing the $\pi(z)$. Even if the problem is hard to solve the $ExpressedOpinion$ performs very well and close to the $Greedy$ algorithm. This happens because it chooses to add an edge based on the expressed opinions and not the reduction of $\pi(z)$ if the edge is added, thus, avoiding computing the quantity of the inverse matrix.
\\
\\
We continued by adopting acceptance probabilities in our heuristics. We measured the mean probability of the edges selected by the heuristics that do and do not consider the acceptance probabilities. In this case we wanted to set an upper bound for the acceptance probabilities with the $maxProb$ algorithm and see if the edited heuristics reach it. We then confirmed that the edited heuristics have a higher mean probability when adding an edge. 

\clearpage

\noindent Finally we compared the reduction of the polarization index with both versions of the heuristics. We observed that the ones that consider acceptance probabilities might have smaller reductions in the $\pi(z)$. This created the trade-off between adding the best candidates to reduce the polarization without knowing if they will be accepted or selecting edges that will most likely be accepted but not have the greatest effect on reducing the $\pi(z)$.




